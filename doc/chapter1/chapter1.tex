\newpage
\chapter{Theory}

\section{Magnetodynamic equation}

\noindent 3D Cartesian coordinate system $(x,y,z)$:

\begin{equation} \label{eq:1}
\nabla \times \left( \frac{1}{\mu} \left( \nabla \times \bm{A} \right) \right) + \gamma \frac{\partial \bm{A}}{\partial t} - \gamma \left( \bm{v} \times \left( \nabla \times \bm{A} \right) \right) = \bm{J}
\end{equation}

\noindent Reduction to 2D $(x,y)$:
\begin{align*} 
\bm{A} &= \left(0,0,A_z\right) \quad &\rightarrow \quad A &= A_z \numberthis \label{eq:2} \\
\bm{J} &= \left(0,0,J_z\right)  \quad &\rightarrow \quad J &= J_z \numberthis \label{eq:3} \\
\bm{v} &= \left(v_x,v_y,0\right)  \quad &\rightarrow \quad \bm{v} &= \left(v_x,v_y\right) \numberthis \label{eq:4} \\
\bm{B} &= \nabla \times \bm{A} = \left(B_x,B_y,0\right)  \quad &\rightarrow \quad \bm{B} &= \left(B_x,B_y\right) \numberthis \label{eq:5}
\end{align*}

\noindent For 2D, time-dependent problems are $A, J, \bm{v}$ and  $\bm{B}$ functions of coordinates $(x,y)$ and time $t$. Both $A$ and $J$ are scalar fields $\mathbb{R}^2 \rightarrow \mathbb{R}^1$, $\bm{v}$ and $\bm{B}$ are vector fields $\mathbb{R}^2 \rightarrow \mathbb{R}^2$.

In a linear material, the $\mu$ and the $\gamma$ are dependent only on the coordinate system $(x,y)$, in the non-linear material, the $\mu$ is also dependent on the size of $||\bm{B}||$.

\begin{equation} \label{eq:6}
\nabla = \left( \partial_x, \partial_y \right)
\end{equation}

\noindent Curl of scalar field $\left(\nabla \times A\right)$ is vector field $\left(\partial_y A, -\partial_x A\right)$, so $\mathbb{R}^1 \rightarrow \mathbb{R}^2$, but curl of vector field $\left(\nabla \times \bm{B}\right)$ is scalar field $\left(\partial_x B_y - \partial_y B_x\right)$, so $\mathbb{R}^2 \rightarrow \mathbb{R}^1$.

Divergence of vector field $\left(\nabla \cdot \bm{B}\right)$ is also scalar field $\left(\partial_x B_x + \partial_y B_y\right)$, so $\mathbb{R}^2 \rightarrow \mathbb{R}^1$.

\begin{equation} \label{eq:7}
\nabla \times \left( \frac{1}{\mu} \left( \nabla \times A \right) \right) + \gamma \frac{\partial A}{\partial t} - \gamma \left( \bm{v} \times \left( \nabla \times A \right) \right) = J
\end{equation}

\section{Boundary conditions}
\noindent Dirichlet condition:
\begin{equation} \label{eq:8} 
A\bigg\rvert_{\partial \Omega_1} = f_1\left(x,y\right)
\end{equation}

\noindent Neumann condition:
\begin{equation} \label{eq:9} 
\partial_{\bm{n}} A\bigg\rvert_{\partial \Omega_2} = f_2\left(x,y\right)
\end{equation}

\noindent For our model, both $f_1$ and $f_2$ are equal to zero.

\section{Initial condition}

\begin{equation} \label{eq:10} 
A\bigg\rvert_{t=0} = 0
\end{equation}

\section{Force}

\begin{equation} \label{eq:11} 
T_{ij} = \frac{1}{\mu_0} \left(B_i B_j - \frac{1}{2} B^2 \delta_{ij}\right)
\end{equation}

\begin{equation} \label{eq:12} 
\mathbb{T} = \frac{1}{\mu_0} \begin{pmatrix}
B_x B_x & B_x B_y \\
B_y B_x & B_y B_y \\
\end{pmatrix} - \frac{1}{2\mu_0} \begin{pmatrix}
B_x^2 + B_y^2 & 0 \\
0 & B_x^2 + B_y^2 \\
\end{pmatrix}
\end{equation}

\begin{equation} \label{eq:13} 
\bm{F}\left(x,y\right) = L \oint_{\partial \Omega} \mathbb{T} \bm{n} \,\mathrm{d}l = L \int_{\Omega} \left( \nabla \cdot \mathbb{T} \right) \,\mathrm{d}S
\end{equation}

\noindent Variable L is the length of the model into the third dimension $z$.

\begin{align*}
\nabla \cdot \mathbb{T} &= \frac{1}{\mu_0} \begin{pmatrix} \frac{1}{2} \partial_x \left(B_x^2 - B_y^2 \right) + \partial_y \left( B_y B_x \right) \\ \frac{1}{2} \partial_y \left(B_y^2 - B_x^2 \right) + \partial_x \left( B_x B_y \right) \\ \end{pmatrix} \\
&= \frac{1}{\mu_0} \begin{pmatrix} \partial_x B_x B_x - \partial_x B_y B_y + \partial_y B_y B_x + \partial_y B_x B_y \\ \partial_y B_y B_y - \partial_y B_x B_x + \partial_x B_x B_y + \partial_x B_y B_x \\ \end{pmatrix} \numberthis \label{eq:14} \\
\end{align*}

\section{Permanent magnets}

\begin{equation} \label{eq:15}
\nabla \times \left( \frac{1}{\mu} \left( \nabla \times A \right) - \frac{1}{\mu} \bm{B_r} \right) + \gamma \frac{\partial A}{\partial t} - \gamma \left( \bm{v} \times \left( \nabla \times A \right) \right) = J
\end{equation}

\noindent Permanent magnets are usually considered as a source of magnetic field, so it is better to move them to the right hand side of the equation.

\begin{equation} \label{eq:16}
\nabla \times \left( \frac{1}{\mu} \left( \nabla \times A \right) \right) + \gamma \frac{\partial A}{\partial t} - \gamma \left( \bm{v} \times \left( \nabla \times A \right) \right) = J + \frac{1}{\mu} \left( \nabla \times \bm{B_r} \right)
\end{equation}