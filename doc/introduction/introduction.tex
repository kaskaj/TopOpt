%\clearpage
\chapter*{Úvod}
\addcontentsline{toc}{chapter}{Úvod} %prida uvod do obsahu

\noindent Ačkoliv poznatky ohledně principů reluktance u synchronních strojů s vyniklými póly sahají až do počátků 20. století, první praktická využití reluktančních strojů se objevila až v 60. a 70. letech 20. století s rozvojem frekvenčních měničů. Tyto motory byly řízeny skalárně a většinou tak ke své funkci vyžadovaly i klec nakrátko. Další vývoj těchto strojů přišel jednak s modernějšími algoritmy, které byly schopné stroje dobře řídit i bez použití této klece, jednak s kolísajícími cenami vzácných zemin využívaných pro permanentní magnety. \cite{redisc}

Výhoda absence rotorového vinutí, případně permanentních magnetů je však u těchto strojů často vyvažována poměrně nízkými hodnotami účinnosti a účiníku a vysokým zvlněním momentu. Analyticko-empirické metody návrhu těchto strojů, především potom jejich rotorů, však často nejsou vždy schopny tyto nároky na provozní parametry splnit. S příchodem sofistikovaných multikriteriálních optimalizačních nástrojů a růstem výkonu výpočetní techniky se objevila další možnost jak tyto náročné úlohy řešit. Prvním úkolem práce tak je provedení rešerše dnes dostupných a používaných nástrojů tvarové a~topologické optimalizace reluktančních strojů. Jsou zde nejprve spíše obecněji vysvětleny principy multikriteriální optimalizace a~dělení jednotlivých algoritmů, dále jsou popsány využívané náhradní modely, které mohou při optimalizaci ušetřit značný čas a potřebný výpočetní výkon a zmíněny jsou i možnosti formalizace geometrie. V dalších podkapitolách, dělených podle náhradních modelů, případně využitých optimalizačních algoritmů, už jsou popsány konkrétní příklady optimalizací v literatuře.

Druhá kapitola je věnována definici řešeného problému, tedy nahrazení rotoru malého asynchronního stroje rotorem reluktančním. Jsou zde popsány základní vlastnosti a rozměry tohoto stroje a následuje popis parametrizace zvolené topologie izolačních bariér a dovolených hodnot a mezí. Třetí kapitola je potom věnována popisu matematického modelu, rozebírány jsou zde řešené rovnice elektromagnetického pole včetně okrajových podmínek, výpočet točivého momentu, ztráty v železe a účiník stroje.

Poslední kapitola se věnuje samotné optimalizaci reluktančního rotoru, je zde blíže popsán použitý genetický algoritmus NSGA-II, jeho implementace do problému a dále popsán průběh optimalizace rotoru ve dvou variantách se třemi a čtyřmi izolačními bariérami. Pro každou variantu jsou dle zvolených kritérií vybrány dva stroje a ty dále porovnávány z hlediska geometrie a provozních parametrů. Následně je provedena i citlivostní analýza, jejímž úkolem je zjistit robustnost jednotlivých strojů. Nakonec je navrhnut experiment, při kterém by byl původní asynchronní stroj porovnán se strojem novým, reluktančním. Pro dva nejlepší rotory jsou v přílohách vytvořeny výrobní výkresy.
