\chapter*{Appendix}
\addcontentsline{toc}{chapter}{Appendix}
\section*{Weak formulation}
\noindent For $\gamma = 0$.

\begin{equation} \label{eq:ap1}
\nabla \times \left( \frac{1}{\mu} \left( \nabla \times A \right) \right) = J
\end{equation}
\noindent The $\theta$ is test function.

\begin{equation} \label{eq:ap2}
\int_{\Omega} \nabla \times \left( \frac{1}{\mu} \left( \nabla \times A \right) \right) \theta \, \mathrm{d}S = \int_{\Omega} J \theta \, \mathrm{d}S
\end{equation}

\begin{align*} 
\nabla \times A &= \left(\partial_y A, -\partial_x A\right) = \bm{B} \\
\nabla \times \bm{B} &= \partial_x B_y - \partial_y B_x = - \partial_x \partial_x A - \partial_y \partial_y A  = -\Delta A \numberthis \label{eq:ap3}
\end{align*}

\begin{equation} \label{eq:ap4}
\int_{\Omega} - \frac{1}{\mu} \Delta A \theta \, \mathrm{d}S = \int_{\Omega} J \theta \, \mathrm{d}S
\end{equation}

\noindent Green's first identity:

\begin{equation} \label{eq:ap5}
\int_{\Omega} \Delta F \theta \, \mathrm{d}S + \int_{\Omega} \nabla F \nabla \theta \, \mathrm{d}S = \int_{\partial\Omega} \left(\nabla F \bm{n} \right) \theta \mathrm{d}l
\end{equation}

\begin{equation} \label{eq:ap5}
\int_{\Omega} - \frac{1}{\mu} \Delta A \theta \, \mathrm{d}S = \int_{\Omega} \frac{1}{\mu} \nabla A \nabla \theta \, \mathrm{d}S - \int_{\partial\Omega_1} \frac{1}{\mu} \left(\nabla A \bm{n} \right) \theta \mathrm{d}l - \int_{\partial\Omega_2} \frac{1}{\mu} \left(\nabla A \bm{n} \right) \theta \mathrm{d}l
\end{equation}

$\theta \in H^1 : \theta \rvert_{\partial\Omega_1} = 0, \theta \rvert_{\partial\Omega_2} = 0 \rightarrow \int_{\partial\Omega_1} \frac{1}{\mu} \left(\nabla A \bm{n} \right) \theta \mathrm{d}l = 0, \int_{\partial\Omega_2} \frac{1}{\mu} \left(\nabla A \bm{n} \right) \theta \mathrm{d}l = 0$
 
\section*{Discretization}

\begin{align*}
A\left(x,y,t\right) \approx \sum_{n=1}^{N} A^n \left(t\right) \lambda^n \left(x,y\right) \numberthis \label{eq:ap6} \\
J\left(x,y,t\right) \approx \sum_{n=1}^{N} J^n \left(t\right) \lambda^n \left(x,y\right) \numberthis \label{eq:ap7} \\
\theta\left(x,y,t\right) \approx \sum_{n=1}^{N} \theta^n \left(t\right) \lambda^n \left(x,y\right) \numberthis \label{eq:ap8} \\
\end{align*}

\begin{equation} \label{eq:ap9}
\int_{T_k} \frac{1}{\mu} \nabla \left( \sum_{n=1}^{N} A^n \lambda^n \right) \nabla \left( \sum_{n=1}^{N} \theta^n \lambda^n \right) \, \mathrm{d}S = \int_{T_k} \sum_{n=1}^{N} J^n \lambda^n \sum_{n=1}^{N} \theta^n \lambda^n \, \mathrm{d}S
\end{equation}

\noindent $T_k$ is one discrete element of geometry, in our case it is a triangle.

\begin{align*} 
\sum_{n=1}^{N} \int_{T_k} \frac{1}{\mu} \nabla \left( A^n \lambda^n \right) \nabla \left( \theta^m \lambda^m \right) \, \mathrm{d}S &= \sum_{n=1}^{N} \int_{T_k} J^n \lambda^n \theta^m \lambda^m \, \mathrm{d}S \\
\sum_{n=1}^{N} A^n \int_{T_k} \frac{1}{\mu} \nabla  \lambda^n  \nabla \lambda^m \, \mathrm{d}S &= \sum_{n=1}^{N} J^n \int_{T_k} \lambda^n \lambda^m \, \mathrm{d}S \numberthis \label{eq:ap10}
\end{align*}

\noindent We can put $A^n$ and $J^n$ out of $\nabla$ and out of $\int_{T_k}$, because they no longer depend on coordinate system.

\begin{align*} 
S &= \sum_{n=1}^{N} \int_{T_k} \frac{1}{\mu} \nabla  \lambda^n  \nabla \lambda^m \, \mathrm{d}S  \\
M &= \sum_{n=1}^{N} \int_{T_k} \lambda^n \lambda^m \, \mathrm{d}S
\end{align*}

\begin{equation} \label{eq:ap11}
S A  =  M J \rightarrow A = S M J
\end{equation}

\noindent If we discretize the problem with triangular elements and select the first-order polynomial (linear function), as an approximation of the scalar field A, we get three basis functions for each triangle. The coordinates $\left(r,s\right)$ correspond to the reference triangle with vertices $V_1 = (0,0), V_2 = (0,1)$ and $V_3 = (1,1)$ numbered in counter-clockwise direction.

\begin{align*}
\beta_1 &= 1 - r - s \numberthis \label{eq:ap12} \\
\beta_2 &= r \numberthis \label{eq:ap13} \\
\beta_3 &= s \numberthis \label{eq:ap14}
\end{align*}

\noindent If we differentiate them in respect to coordinates, we get:

\begin{equation} \label{eq:ap15}
\nabla \beta = \begin{pmatrix} \partial_r \beta_1 & \partial_r \beta_2 & \partial_r \beta_3 \\ \partial_s \beta_1 & \partial_s \beta_2 & \partial_s \beta_3 \\ \end{pmatrix} =  \begin{pmatrix} -1 & 1 & 0 \\ -1 & 0 & 1 \\ \end{pmatrix}
\end{equation}

\noindent The matrices, fields and functions of the reference triangle must then be transformed into our coordinate system $(r,s) \rightarrow (x,y)$ and put into right place. Vertices of the triangle in $(x,y)$ coordinates are $V_1 = (x_1,y_1), V_2 = (x_2,y_2)$ and $V_3 = (x_3,y_3)$.

\begin{equation} \label{eq:ap16}
\Phi \begin{pmatrix} x \\ y \\ \end{pmatrix} = \begin{pmatrix} x_1 \\ y_1 \\ \end{pmatrix} + \begin{pmatrix} x_2 - x_1 & x_3 - x_1  \\ y_2 - y_1 & y_3 - y_1 \\ \end{pmatrix} \begin{pmatrix} r \\ s \\ \end{pmatrix}
\end{equation}

\noindent The $\Phi$ is the transformation function. We can use it to transform basis functions of reference triangle $T_r$ to basis functions of the n-th triangle $T_k$.

\begin{equation} \label{eq:ap18}
\lambda\left(x,y\right) = \left( \beta \circ \Phi^{-1}\right) \left(x,y\right)
\end{equation}

\noindent We will also need the inverse Jacobian matrix and the “Jacobian”, ie the determinant of the Jacobian matrix.

\begin{equation} \label{eq:ap19}
\vert \mathrm{det} \left(\nabla \Phi \right) \vert = \left| \mathrm{det} \begin{pmatrix} x_2 - x_1 & x_3 - x_1  \\ y_2 - y_1 & y_3 - y_1 \\ \end{pmatrix} \right|
\end{equation} 

\begin{equation} \label{eq:ap20}
\nabla \Phi^{-1} = \frac{1}{\vert \mathrm{det} \left(\nabla \Phi \right) \vert} \begin{pmatrix} y_3 - y_1  &    -\left(x_3 - x_1\right) \\ -\left( y_2 - y_1 \right) & x_2 - x_1 \\ \end{pmatrix}
\end{equation}

\noindent Mass matrix:

\begin{align*}
M_{T_k} &= \int_{T_k} \lambda^n \lambda^m \, \mathrm{d}S =  \int_{T_k} \left( \beta^n \circ \Phi^{-1}\right) \left( \beta^m \circ \Phi^{-1}\right) \, \mathrm{d}S \\
&= \vert \mathrm{det} \left(\nabla \Phi \right) \vert \int_{T_r} \beta^n \beta^m \, \mathrm{d}S \numberthis \label{eq:ap21}
\end{align*}

\begin{equation} \label{eq:ap22}
M_{T_k} = \vert \mathrm{det} \left(\nabla \Phi \right) \vert \begin{pmatrix} \int \beta_1 \beta_1 \, \mathrm{d}S & \int \beta_1 \beta_2 \, \mathrm{d}S  & \int \beta_1 \beta_3 \, \mathrm{d}S  \\ \int \beta_2 \beta_1 \, \mathrm{d}S & \int \beta_2 \beta_2 \, \mathrm{d}S  & \int \beta_2 \beta_3 \, \mathrm{d}S \\ \int \beta_3 \beta_1 \, \mathrm{d}S & \int \beta_3 \beta_2 \, \mathrm{d}S & \int \beta_3 \beta_3 \, \mathrm{d}S \\ \end{pmatrix}
\end{equation}

\begin{equation} \label{eq:ap23}
M = \sum_{n=1}^{N} \vert \mathrm{det} \left(\nabla \Phi \right) \vert \begin{pmatrix} \frac{1}{12} & \frac{1}{24}  & \frac{1}{24}  \\ \frac{1}{24} & \frac{1}{12}  & \frac{1}{24} \\ \frac{1}{24}  & \frac{1}{24} & \frac{1}{12} \\ \end{pmatrix}
\end{equation}

\noindent Stiffness matrix:

\begin{align*}
S_{T_k} &= \int_{T_k} \frac{1}{\mu} \nabla  \lambda^n  \nabla \lambda^m \, \mathrm{d}S = \int_{T_k} \frac{1}{\mu} \nabla  \left( \beta^n \circ \Phi^{-1}\right)  \nabla \left( \beta^m \circ \Phi^{-1}\right) \, \mathrm{d}S  \\
&= \int_{T_k} \frac{1}{\mu} \partial_x \left( \beta^n \circ \Phi^{-1}\right) \partial_x \left( \beta^m \circ \Phi^{-1}\right) \, \mathrm{d}S +  \int_{T_k} \frac{1}{\mu}  \partial_y \left( \beta^n \circ \Phi^{-1}\right)  \partial_y \left( \beta^m \circ \Phi^{-1}\right) \, \mathrm{d}S \\
&= S_{T_k,x} + S_{T_k,y} \numberthis \label{eq:ap24}
\end{align*}

\begin{align*}
S_{T_k,x} &= \int_{T_k} \frac{1}{\mu} \left( \partial_x \beta^n \left( \Phi^{-1} \right) \right) \left( \partial_x \Phi^{-1} \right) \left( \partial_x \beta^n \left( \Phi^{-1} \right) \right) \left( \partial_x \Phi^{-1} \right) \, \mathrm{d}S \\
&= \frac{1}{\mu} \vert \mathrm{det} \left(\nabla \Phi \right) \vert \left(1,0\right)  \int_{T_r} \nabla \Phi^{-1}  \nabla \beta^{\top} \nabla \Phi^{-\top}  \nabla \beta \,\mathrm{d}S  \numberthis \label{eq:ap25}
\end{align*}

\begin{align*} 
S = &\sum_{n=1}^{N} \frac{1}{2 \mu} \vert \mathrm{det} \left(\nabla \Phi \right) \vert  \left(1,0\right) \nabla \Phi^{-1} \nabla \beta^{\top}  \nabla \Phi^{-\top} \nabla \beta \\
&+ \sum_{n=1}^{N}  \frac{1}{2 \mu} \vert \mathrm{det} \left(\nabla \Phi \right) \vert  \left(0,1\right)\nabla \Phi^{-1} \nabla \beta^{\top} \nabla \Phi^{-\top} \nabla \beta \numberthis \label{eq:ap26}
\end{align*}

\noindent If we want to get the vector field $\bm{B}$ from the scalar field $A$, we need a “curl matrix”.

\begin{equation} \label{eq:ap27}
\bm{B} = \begin{pmatrix} B_x \\ B_y \end{pmatrix} = \begin{pmatrix} \partial_y A \\ -\partial_x A \end{pmatrix}
\end{equation}

\begin{equation} \label{eq:ap28}
\int_{\Omega} B_x \theta \,\mathrm{d}S =  \int_{\Omega} \partial_y A \theta \,\mathrm{d}S
\end{equation}

\noindent Integration by parts:

\begin{equation} \label{eq:ap29}
\int_{\Omega} \partial F \theta \,\mathrm{d}S = \int_{\partial\Omega}  F \theta \bm{n} \mathrm{d}l  - \int_{\Omega} F \partial \theta \,\mathrm{d}S
\end{equation}

\begin{equation} \label{eq:ap30}
\int_{\Omega} B_x \theta \,\mathrm{d}S = \int_{\Omega} \partial_y A \theta \,\mathrm{d}S = \int_{\partial\Omega} A \theta \bm{n} \,\mathrm{d}l - \int_{\Omega} A \partial_y \theta \,\mathrm{d}S
\end{equation}

\noindent Constraints for $\theta$ ...

\begin{equation} \label{eq:ap40}
\int_{\Omega} \sum_{n=1}^{N} B_x^n \lambda^n \sum_{n=1}^{N} \theta^n \lambda^n \,\mathrm{d}S = -\int_{\Omega} \sum_{n=1}^{N} A^n \lambda^n \partial_y \left(  \sum_{n=1}^{N} \theta^n \lambda^n \right) \,\mathrm{d}S 
\end{equation}

\begin{equation} \label{eq:ap41}
\sum_{n=1}^{N} B_x^n \int_{\Omega} \lambda^n \lambda^m \,\mathrm{d}S =  -\sum_{n=1}^{N}  A^n \int_{\Omega} \lambda^n  \partial_y \lambda^m  \,\mathrm{d}S 
\end{equation}

\begin{equation} \label{eq:ap42} M B_x = - C_y A \rightarrow B_x = - M C_y A
\end{equation}

\noindent “Curl matrix”:

\begin{align*} C_{T_k,y} &=  \int_{T_k} \lambda^n \partial_y \lambda^m  \,\mathrm{d}S = \int_{T_k} \left( \beta^n \circ \Phi^{-1}\right) \partial_y \left( \beta^m \circ \Phi^{-1}\right) \,\mathrm{d}S \\
&= \int_{T_k} \left( \beta^n \circ \Phi^{-1}\right)   \left( \partial_y \beta^n \left(\Phi^{-1}\right) \right) \left( \partial_x \Phi^{-1} \right) \,\mathrm{d}S \\
&= \vert \mathrm{det} \left(\nabla \Phi \right) \vert  \left(1,0\right) \int_{T_r} \beta \nabla \Phi^{-\top} \nabla \beta \,\mathrm{d}S \numberthis \label{eq:ap44}
\end{align*}

\noindent We got $\frac{1}{6} = \frac{1}{2} \cdot \frac{1}{3}$, because $\int \beta \, \mathrm{d}S = \frac{1}{3}$.

\begin{align*} 
C_x &= \sum_{n=1}^{N} \frac{1}{6} \vert \mathrm{det} \left(\nabla \Phi \right) \vert \left(0,1\right) \nabla \Phi^{-\top} \nabla \beta \\
C_y &= - \sum_{n=1}^{N} \frac{1}{6} \vert \mathrm{det} \left(\nabla \Phi \right) \vert \left(1,0\right) \nabla \Phi^{-\top} \nabla \beta \numberthis \label{eq:ap45} 
\end{align*}

\noindent This matrix is also used when permanent magnets are considered as a another magnetic field source. Magnets are represented by their remanent magnetic field density $\bm{B_r}$ and in the magnetic field equation they appear as $\nabla \times \bm{B_r}$. The right hand side of the equation with both coils and permanent magnets is then: $M J + \left( C_y B_{r_x} + C_x B_{r_y} \right)$.


















%\noindent LocalMatrices.m
%
%\verb|edet| = $\vert \mathrm{det} \left(\nabla \Phi \right) \vert$
%
%\verb|dFinv| = $\nabla \Phi^{-1}$
%
%\verb|dphi| = $\nabla \Phi^{-\top} \nabla \beta$
%
%\verb|slocx|  = $3C_{T_k,x}$ = \verb|1/2 *  dphi(1,:) * edet|                        
%
%\verb|slocy|  = $3C_{T_k,y}$ = \verb|1/2 *  dphi(2,:) * edet|                       
%
%\verb|slocxx| = $\mu S_{T_k,x}$ = \verb|1/2 * dphi(1,:)' * dphi(1,:) * edet|
%
%\verb|slocyy| = $\mu S_{T_k,y}$ = \verb|1/2 * dphi(2,:)' * dphi(2,:) * edet|
%
%\verb|mloc| = $M_{T_k}$ = \verb|[1 1/2 1/2;1/2 1 1/2;1/2 1/2 1]/12 * edet|
%
%\verb|clocx|  = $C_{T_k,x}$ =  \verb|[slocx;slocx;slocx]/3|                   
%
%\verb|clocy|  = $C_{T_k,y}$ =  \verb|[slocy;slocy;slocy]/3|                   

