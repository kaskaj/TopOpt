\chapter*{Appendix}
\addcontentsline{toc}{chapter}{Appendix}
\section*{Weak formulation}
\noindent For $\gamma = 0$.

\begin{equation} \label{eq:ap1}
\nabla \times \left( \frac{1}{\mu} \left( \nabla \times A \right) \right) = J
\end{equation}
\noindent The $\theta$ is test function.

\begin{equation} \label{eq:ap2}
\int_{\Omega} \nabla \times \left( \frac{1}{\mu} \left( \nabla \times A \right) \right) \theta \mathrm{d}S = \int_{\Omega} J \theta \mathrm{d}S
\end{equation}

\begin{align*} 
\nabla \times A &= \left(\partial_y A, -\partial_x A\right) = \bm{B} \\
\nabla \times \bm{B} &= \partial_x B_y - \partial_y B_x = - \partial_x \partial_x A - \partial_y \partial_y A  = -\Delta A \numberthis \label{eq:ap3}
\end{align*}

\begin{equation} \label{eq:ap4}
\int_{\Omega} - \frac{1}{\mu} \Delta A \theta \mathrm{d}S = \int_{\Omega} J \theta \mathrm{d}S
\end{equation}

\noindent Green's first identity:

\begin{equation} \label{eq:ap5}
\int_{\Omega} \Delta F \theta \mathrm{d}S + \int_{\Omega} \nabla F \nabla \theta \mathrm{d}S = \int_{\partial\Omega} \left(\nabla F \bm{n} \right) \theta \mathrm{d}l
\end{equation}

\begin{equation} \label{eq:ap5}
\int_{\Omega} - \frac{1}{\mu} \Delta A \theta \mathrm{d}S = \int_{\Omega} \frac{1}{\mu} \nabla A \nabla \theta \mathrm{d}S - \int_{\partial\Omega1} \frac{1}{\mu} \left(\nabla A \bm{n} \right) \theta \mathrm{d}l - \int_{\partial\Omega2} \frac{1}{\mu} \left(\nabla A \bm{n} \right) \theta \mathrm{d}l
\end{equation}

$\theta \in f \left(\Omega\right) : \theta \rvert_{\partial\Omega1} = 0, \theta \rvert_{\partial\Omega2} = 0 \rightarrow \int_{\partial\Omega1} \frac{1}{\mu} \left(\nabla A \bm{n} \right) \theta \mathrm{d}l = 0, \int_{\partial\Omega2} \frac{1}{\mu} \left(\nabla A \bm{n} \right) \theta \mathrm{d}l = 0$
 
\noindent Discretization

\begin{align*}
A\left(x,y,t\right) \approx \sum_{n=1}^{N} A^n \left(t\right) \lambda^n \left(x,y\right) \numberthis \label{eq:ap6} \\
J\left(x,y,t\right) \approx \sum_{n=1}^{N} J^n \left(t\right) \lambda^n \left(x,y\right) \numberthis \label{eq:ap7} \\
\theta\left(x,y,t\right) \approx \sum_{n=1}^{N} \theta^n \left(t\right) \lambda^n \left(x,y\right) \numberthis \label{eq:ap8} \\
\end{align*}

\begin{equation} \label{eq:ap9}
\int_{T_n} \frac{1}{\mu} \nabla \left( \sum_{n=1}^{N} A^n \lambda^n \right) \nabla \left( \sum_{n=1}^{N} \theta^n \lambda^n \right) \mathrm{d}S = \int_{T_n} \sum_{n=1}^{N} J^n \lambda^n \sum_{n=1}^{N} \theta^n \lambda^n \mathrm{d}S
\end{equation}

\noindent $T_n$ is one descrete element of geometry, in our case it is a triangle.

\begin{align*} 
\sum_{n=1}^{N} \int_{T_n} \frac{1}{\mu} \nabla \left( A^n \lambda^n \right) \nabla \left( \theta^m \lambda^m \right) \mathrm{d}S &= \sum_{n=1}^{N} \int_{T_n} J^n \lambda^n \theta^m \lambda^m \mathrm{d}S \\
\sum_{n=1}^{N} A^n \int_{T_n} \frac{1}{\mu} \nabla  \lambda^n  \nabla \lambda^m \mathrm{d}S &= \sum_{n=1}^{N} J^n \int_{T_n} \lambda^n \lambda^m \mathrm{d}S \numberthis \label{eq:ap10}
\end{align*}

\noindent We can put $A^n$ and $J^n$ out of $\nabla$ and out of $\int_{T_n}$, because they no longer depend on coordinate system.

\begin{align*} 
S &= \sum_{n=1}^{N} \int_{T_n} \frac{1}{\mu} \nabla  \lambda^n  \nabla \lambda^m \mathrm{d}S  \\
M &= \sum_{n=1}^{N} \int_{T_n} \lambda^n \lambda^m \mathrm{d}S
\end{align*}

\begin{equation} \label{eq:ap11}
S A  =  M J \rightarrow A = S^\top M J
\end{equation}

\noindent If we discretize the problem with triangular elements and select the first-order polynomial (linear function), as an approximation of the scalar field A, we get three basis functions for each triangle. The coordinates $\left(r,s\right)$ correspond to the reference triangle with vertices $V_1 = (0,0), V_2 = (0,1)$ and $V_3 = (1,1)$ numbered in counter-clockwise direction.

\begin{align*}
\beta_1 &= 1 - r - s \numberthis \label{eq:ap12} \\
\beta_2 &= r \numberthis \label{eq:ap13} \\
\beta_3 &= s \numberthis \label{eq:ap14}
\end{align*}

\noindent If we differentiate them in respect to coordinates, we get:

\begin{equation} \label{eq:ap15}
\partial \beta = \begin{pmatrix} \partial_r \beta_1 & \partial_r \beta_2 & \partial_r \beta_3 \\ \partial_s \beta_1 & \partial_s \beta_2 & \partial_s \beta_3 \\ \end{pmatrix} =  \begin{pmatrix} -1 & 1 & 0 \\ -1 & 0 & 1 \\ \end{pmatrix}
\end{equation}

\noindent The matrices, fields and functions of the reference triangle must then be transformed into our coordinate system $(r,s) \rightarrow (x,y)$ and put into right place. Vertices of the triangle in $(x,y)$ coordinates are $V_1 = (x_1,y_1), V_2 = (x_2,y_2)$ and $V_3 = (x_3,y_3)$.

\begin{equation} \label{eq:ap16}
\Phi \begin{pmatrix} x \\ y \\ \end{pmatrix} = \begin{pmatrix} x_1 \\ y_1 \\ \end{pmatrix} + \begin{pmatrix} x_2 - x_1 & x_3 - x_1  \\ y_2 - y_1 & y_3 - y_1 \\ \end{pmatrix} \begin{pmatrix} r \\ s \\ \end{pmatrix}
\end{equation}

\noindent The $\Phi$ is the transformation function. We can use it to transform basis functions of reference triangle $T_r$ to basis functions of the n-th triangle $T_n$.

\begin{equation} \label{eq:ap18}
\lambda\left(x,y\right) = \left( \beta \circ \Phi^{-1}\right) \left(x,y\right)
\end{equation}

\noindent We will also need the inverse Jacobian matrix and the “Jacobian”, ie the determinant of the Jacobian matrix.

\begin{equation} \label{eq:ap19}
\vert \mathrm{det} \left(\nabla \Phi \right) \vert = \left| \mathrm{det} \begin{pmatrix} x_2 - x_1 & x_3 - x_1  \\ y_2 - y_1 & y_3 - y_1 \\ \end{pmatrix} \right|
\end{equation} 

\begin{equation} \label{eq:ap20}
\left(\nabla \Phi \right)^{-1} = \frac{1}{\vert \mathrm{det} \left(\nabla \Phi \right) \vert} \begin{pmatrix} y_3 - y_1  &    -\left(x_3 - x_1\right) \\ -\left( y_2 - y_1 \right) & x_2 - x_1 \\ \end{pmatrix}
\end{equation}

\noindent Mass matrix:

\begin{align*}
M_{T_n} &= \int_{T_n} \lambda^n \lambda^m \mathrm{d}S =  \int_{T_n} \left( \beta \circ \Phi^{-1}\right)^n \left( \beta \circ \Phi^{-1}\right)^m \mathrm{d}S \\
&= \vert \mathrm{det} \left(\nabla \Phi \right) \vert \int_{T_r} \beta^n \beta^m \mathrm{d}S
\end{align*}


\begin{align*}
\int_{T_r} \left(1-r-s\right)^2 \mathrm{d}S =  \qquad \int_{T_r} \left(1-r-s\right)r \,\mathrm{d}S = \qquad \int_{T_r} \left(1-r-s\right)s \,\mathrm{d}S = \\
\int_{T_r} s^2 \,\mathrm{d}S =  \qquad \int_{T_r} r^2 \,\mathrm{d}S =  \qquad \int_{T_r} r \,\mathrm{d}S = \qquad \int_{T_r} s \,\mathrm{d}S = \qquad \int_{T_r} rs \,\mathrm{d}S =  \\
\end{align*}


%mloc   = edet*[1/12 1/24 1/24;1/24 1/12 1/24;1/24 1/24 1]/12;